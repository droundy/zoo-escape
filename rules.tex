\documentclass[12pt,twocolumn]{article}

\usepackage[paperwidth=9.63in,paperheight=6.88in,width=9.0in,height=6.0in]{geometry}
\usepackage{color}
\usepackage{parskip}

\pagestyle{empty}
\thispagestyle{empty}

\begin{document}
  \color{black}
\emph{Oh no! The animals are escaping from their enclosures at night! Each night more
animals escape.  Fortunately, each night the zookeepers get better at catching
them and returning them to their homes.}

Begin with just one escaped animal, and increase the number of escaped animals
but one each time you play.  Select the animals at random from among the adult
animals, and place them in their enclosures.  Decide how fast each animal can
move.  For perspective, a cheetah can move very fast, 3 squares in a single
turn, while a baby animal can only move one square per turn.

\paragraph{Setup}
Each player chooses a zookeeper token and places it on a zookeeper starting
square.  Each player draws one more zookeeper card than the number of escaped
animals, and places them face up in front of them.

\paragraph{Play}
Each round of play consists of the following sequence:
\begin{enumerate}
\item Draw an animal card and place it in the animal card discard pile.  Move
the animal in the specified direction.  An animal moving fast will move as far
as it can in that direction.  There are two special animal cards:

\textbf{Baby!} If an adult draws the baby card, the baby of the same species is
placed on the same square.  If the animal has already had a baby or is a baby
itself, it simply doesn't move.

\textbf{Nap.} The animal doesn't move this turn.
\item Repeat for each animal.
\item Zookeeper selects one card to discard, and draws a new card to replace it.

\textbf{Go to the bathroom.}
The \textbf{Go to the bathroom} card may only be discarded when on a restroom
square.  I a zookeeper who is not at a bathroom has only \textbf{Go to the
bathroom} cards in their hand, they neither discard or draw.
\item  Zookeeper selects one of the moves in front of them, and moves their
token accordingly.

\textbf{Go to the bathroom.}  If a player has a \textbf{Go to the bathroom}
card, they may choose to move one space in the direction of the nearest
bathroom.
\item Repeat for each zookeeper.
\end{enumerate}
When the animal or zookeeper draw pile is empty, shuffle the
corresponding discard pile to create a new draw pile.

\paragraph{End of game}
If at any point an animal shares a square with a zookeeper that animal is
captured and returned to its enclosure after an appropriate amount of cuddles.

The game ends when all the animals have been captured, and the next game will
begin with one more animal and one more zookeeper card for each zookeeper.



\end{document}
